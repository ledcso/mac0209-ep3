\documentclass{article}

% Language setting
% Replace `english' with e.g. `spanish' to change the document language
\usepackage[portuguese]{babel}

% Set page size and margins
% Replace `letterpaper' with `a4paper' for UK/EU standard size
\usepackage[letterpaper,top=2cm,bottom=2cm,left=3cm,right=3cm,marginparwidth=1.75cm]{geometry}

% Useful packages
\usepackage{amsmath}
\usepackage{graphicx}
\usepackage[colorlinks=true, allcolors=blue]{hyperref}

\title{Relatório do EP 3 de MAC0209}
\author{Identificação dos componentes do grupo}

\begin{document}
\maketitle


\begin{abstract}
Resumo do relatório.

O vídeo do EP deve ser \href{YOUTUBE}{referenciado aqui}.

\end{abstract}

\newpage

\tableofcontents

\newpage

\section{Introdução}

Apresente uma introdução ao trabalho desenvolvido, fornecendo o contexto e a motivação.

\section{Objetivos}

Apresente o objetivo dessa parte do trabalho. Seja {\bf objetivo e claro}.

\section{Cronograma}

Nesta seção, o grupo deve apresentar a Gantt Chartt de planejamento para o desenvolvimento do EP. Esta seção é a única obrigatória para o EP "Preparação do EP".

\section{Dados e métodos}

Explique os dados usados e os métodos desenvolvidos.

\section{Resultados experimentais}

Apresente os resultados obtidos, Explore tabelas e gráficos ilustrativos. Interprete os resultados e apresente uma visão crítica.

\section{Discussão e Conclusão}

Discuta os principais aspectos do EP desenvolvido e dos resultados obtidos. Apresente a conclusão do trabalho.



\end{document}
